
%% see http://latex-beamer.sourceforge.net/
%% idea contributed by H. Turgut Uyar
%% template based on a template by Till Tantau
%% this template is still evolving - it might differ in future releases!

\documentclass[xcolor={usenames,dvipsnames}]{beamer}
\usepackage{booktabs}


% THEME
% =========================================================
%\usetheme{Boadilla}
\setbeamertemplate{navigation symbols}{}
\setbeamercolor{normal text}{fg=white,bg=black!90}
\setbeamercolor{structure}{fg=white}
\setbeamercolor{item projected}{use=item,fg=white,bg=item.fg!35}
\setbeamercolor*{palette primary}{use=structure,fg=structure.fg}
\setbeamercolor*{palette secondary}{use=structure,fg=structure.fg!95!black}
\setbeamercolor*{palette tertiary}{use=structure,fg=structure.fg!90!black}
\setbeamercolor*{palette quaternary}{use=structure,fg=structure.fg!95!black,bg=black!80}
\setbeamercolor*{framesubtitle}{fg=white}
\setbeamercolor*{block title}{parent=structure,bg=black!60}
\setbeamercolor*{block body}{fg=black,bg=black!10}
\setbeamercolor*{block title alerted}{parent=alerted text,bg=black!15}
\setbeamercolor*{block title example}{parent=example text,bg=black!15}
%\useinnertheme[shadow]{rounded}
% =========================================================

% DEFINE COLORS
% =========================================================
\definecolor{darkred}{RGB}{223,63,00}
\definecolor{brightred}{RGB}{255,127,00}
% =========================================================

% Templates
% =========================================================
%\setbeamertemplate{itemize subitem}[triangle]
% =========================================================

% SET COLORS
% =========================================================
\setbeamercolor{normal text}{fg=white,bg=black!90}
\setbeamercolor{structure}{fg=white}
\setbeamercolor{item projected}{use=item,fg=white,bg=item.fg!35}
\setbeamercolor*{palette primary}{use=structure,fg=structure.fg}
\setbeamercolor*{palette secondary}{use=structure,fg=structure.fg!95!black}
\setbeamercolor*{palette tertiary}{use=structure,fg=structure.fg!90!black}
\setbeamercolor*{palette quaternary}{use=structure,fg=structure.fg!95!black,bg=black!80}
\setbeamercolor*{framesubtitle}{fg=white}
\setbeamercolor*{block title}{parent=structure,bg=black!60}
\setbeamercolor*{block body}{fg=black,bg=black!10}
\setbeamercolor*{block title alerted}{parent=alerted text,bg=black!15}
\setbeamercolor*{block title example}{parent=example text,bg=black!15}
% =========================================================


% FONTS
% =========================================================
\setbeamerfont{alerted text}{series=\bfseries}
% =========================================================


% PACKAGES
% =========================================================
\usepackage[english]{babel}
\usepackage[utf8]{inputenc}
\usepackage{DejaVuSansMono}
\usepackage[T1]{fontenc}
\usepackage{tikz}
% =========================================================

% METADATA
% =========================================================
\title{Concepts of Trust Establishment\\ WS 2023/24}

\subtitle{Seminar}

\author[M. Tschirschnitz]
{
	Maximilian von Tschirschnitz
}

\institute[Chair I20, TUM]
{
	Lehrstuhl f\"ur Sicherheit in der Informatik / I20 \\
	Prof.\ Dr.\ Claudia Eckert\\
	Technische Universität München
}

\date{\today}
% =========================================================

\begin{document}

\begin{frame}
	\titlepage
\end{frame}

\begin{frame}
	\frametitle{What is this seminar about?}

	Grasping the concepts of Trust and how we can establish it (autonomously).
	\hfill
	\begin{itemize}
		\item What does the relation of Trust actually mean?
		\item What established/novel concepts of Trust Establishment are available ?
		\item What Axioms can we choose from to rely upon ? 
	\end{itemize}
\end{frame}

%
\begin{frame}[label=process]
	\frametitle{Process}
	\begin{itemize}
		\item Phase \alert{I}: Select a \alert{topic}
		\item Phase \alert{II}: Find \alert{literature}
		\item Phase \alert{III}: Do your \alert{reading / experiments / programming}
		\item Phase \alert{IV}: \alert{Writing} phase I
		\item Phase \alert{V}: \alert{Peer review}
		\item Phase \alert{VI}: \alert{Writing} phase II
		\item Phase \alert{VII}: Final \alert{talks}
	\end{itemize}
	Exact schedule will be published once list of participants is known.
\end{frame}

\begin{frame}
	\frametitle{Phase \alert{I}}
	\begin{enumerate}
		\item I will provide you with a list of starting points for topics that are of interest for this seminar
		\item You will \alert{choose / propose} your topic and thereby either:
			\begin{itemize}
				\item Model, analyse or compare some existing definitions of, and approaches to Trust.
				\item Conduct a critical analysis of an existing Trust establishment method
				\item Elaborate on novel (cryptographic) approaches (e.g. Homomorphic Time Lock Puzzles)
				\item Categorize existing Trust establishment methods 
				\item Elaborate on concepts of tracking/trusting origin
			\end{itemize}
		\item In all cases, you will put your work into context of existing literature
			\begin{itemize}
				\item e.g at Usenix Security Symposium, S\&P, ACM CCS, NDSS
			\end{itemize}
	\end{enumerate}
\end{frame}

\begin{frame}
	\frametitle{Our Topics of Interest}
	\begin{itemize}
		\item Proximity as Trust Factor
		\item Trustmanagement in Groups/Teams
		\item Survey on Trust establishment in Ad hoc networks
		\item Game Theory applied to Ad Hoc networks
		\item Integrity Codes / Tamper Evident Pairing
		\item Differentiation of available `Web of Trust'/PKI Concepts
		\item Comparison of formal verification approaches
		\item Homomorphic Time Lock Puzzles
		\item Provenance and Dependency Analysis in relation to Authenticity
		\item Password Authenticated Key agreement and Zero Knowledge Proofs
		\item \alert{Or:} Provide me with your own topic proposal and I will consider it
	\end{itemize}
\end{frame}

\begin{frame}
	\frametitle{Registration}
	\begin{itemize}
		\item Registration using the \alert{matching system}
	  	\item Letter of motivation gets preference
	  	\item Email \alert{one paragraph} why you want to do this seminar
	  	\item Your interests/skillset for that course
	  	\item Send \alert{with subject} [CTE] to tschirschnitz@sec.in.tum.de 
	\end{itemize}
\end{frame}

\begin{frame}<1>[label=timeplace]
	\frametitle{Time and Place}
	\begin{center}
		\begin{tabular}{ll}
		        \textbf{When?}  & I pick the slot\\
		        						&\textcircled{1} for Bi-Weekly Meetings during the Semester\\
		        						&\textcircled{2} with the least collisions\\
		        						&\textcircled{3} Physical attendance mandatory!\\
					&\\
					& Talks at the \alert{end} of the semester \\
		\end{tabular}
	\end{center}
\end{frame}

\againframe<2->{timeplace}

\begin{frame}
\frametitle{Grading}
  \begin{tabular}{lrl}
             & \alert{40 \%}   & Final Paper (Content, Style, Language, Scope, \ldots)\\
	     		 & \alert{10 \%}   & Practical application (depends on topic)  \\
             & \alert{10 \%}   & Review      \\
             & \alert{30 \%}   & Presentation (Content, Style, Timeliness, \ldots) \\
             & \alert{10 \%}   & Discussion                                                   \\
    \midrule
	$\Sigma$ & \alert{100 \%}  & Total                                                        \\
  \end{tabular}
\end{frame}

\begin{frame}
	\begin{center}
		{\huge Questions?}

		\vspace{2cm}

		\begin{center}
			Contact me at \\ \texttt{tschirschnitz@sec.in.tum.de}
		\end{center}

		\vspace{1cm}
	\end{center}
\end{frame}


\end{document}
