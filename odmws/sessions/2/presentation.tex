\documentclass{i20lecture}

\subtitle{Session 2}
\usepackage{amssymb}

\newcommand{\done}{\makebox[0pt][l]{$\square$}\raisebox{.15ex}{\hspace{0.1em}$\checkmark$}}
\begin{document}

\frame{\titlepage}

%%%%%%%%%%%%%%%%%%%%%%%%%%%%%%%%%%%%%%%%%%%%%%%%%%%%%%%%%%%%%%%%%%%%%%%%%%%%%%%%%%%%%
\begin{frame}
  \frametitle{Anmerkungen zu den Entwürfen}
  \framesubtitle{Sprache}

  \begin{itemize}
    \item Der Gebrauch eines Kommas vor \emph{\textsc{which}} ist im Englischen nur
    korrekt, wenn der eingeleitete Nebensatz \emph{optional} für den \emph{Sinn} des gesamten
    Satzes ist.
    \begin{itemize}
      \item ``Principia Mathematica'' is one of the most complete works
      (\emph{kein Komma}) which examines different ways to use logical inference
      to obtain meaningful results.
      \item Type-1 hypervisors\emph{,} which are also called bare-metal hypervisors\emph{,}
      execute directly on the hardware without relying on any operating system.
    \end{itemize}
    
    \item Die Verwendung von \emph{\textsc{that}} ist nur bei \emph{nicht-optionalen}
    Nebensätzen zulässig $\Rightarrow$ Vor \emph{\textsc{that}} steht im Englischen
    (fast) nie ein Komma.
    \item Kurzformen wie \emph{\textsc{don't}}, \emph{\textsc{we'll}} und \emph{\textsc{kinda}} haben in einer Seminararbeit
    nichts verloren.
    \item Das Zeichen \texttt{-} ist ein Trenn- oder Bindestrich. Ein
    Gedankenstrich wird in \LaTeX\ durch \texttt{-}\texttt{-} oder
    \texttt{-}\texttt{-}\texttt{-}
    produziert.
  \end{itemize}
\end{frame}

%%%%%%%%%%%%%%%%%%%%%%%%%%%%%%%%%%%%%%%%%%%%%%%%%%%%%%%%%%%%%%%%%%%%%%%%%%%%%%%%%%%%%
\begin{frame}
  \frametitle{Anmerkungen zu den Entwürfen}
  \framesubtitle{Sprache}

  \begin{itemize}
    \item Übermäßiger Gebrauch \emph{parataktischer} Sätze ist zu vermeiden. Hier hilft es,
    die Sätze nach dem ``Behauptung -- Begründung -- Beispiel''-Schema aufzubauen und
    sinnvolle \emph{Konjunktionen} zu verwenden.
    \begin{itemize}
      \item The networking subsystems are often also only implemented partly.
      There might be differences compared to a non-virtualized system. This can
      be detected by malware.
      \item Malware can exploit discrepancies between expected behavior of
      networking hardware and an emulated environment. This is due to the fact
      that contemporary hardware components are increasingly complex,
      consequently imposing the burden of emulating all corner cases correctly
      on implementors. For example, networking hardware emulated by Qemu version
      2.6 lacks support of out-of-band management capabilities like Intel AMT.
    \end{itemize}
    \item Sätze, welche aus diesem Schema fallen und nurmehr \emph{Pauschalaussagen} darstellen
    fallen im Allgemeinen weg.
    \begin{itemize}
      \item Several levels of intermediate representations and transformations
      between those are common to deal with different aspects of reasoning about
      the program.
    \end{itemize}
  \end{itemize}
\end{frame}

%%%%%%%%%%%%%%%%%%%%%%%%%%%%%%%%%%%%%%%%%%%%%%%%%%%%%%%%%%%%%%%%%%%%%%%%%%%%%%%%%%%%%
\begin{frame}
  \frametitle{Anmerkungen zu den Entwürfen}
  \framesubtitle{Form}

  \begin{itemize}
    \item Vom Text \emph{nicht referenzierte} Abbildungen kommen \emph{nicht} vor.
    
    \item \emph{Quellenangaben} stehen am \emph{Ende} des Satzes kurz \emph{vor} dem Punkt, oder direkt
    \emph{hinter} dem \emph{Namen} eines zitierten \emph{Programms}/\emph{Frameworks}.
    
    \item Die \emph{Einträge} im Quellenverzeichnis beinhalten einheitlich mindestens \emph{Autor}, \emph{Titel}, und
    \emph{Jahr}. Abhängig von der Art der Quelle ist ein Link mit Abrufzeitpunkt oder die
    Konferez / Journal vorhanden. Ist diese Information \emph{nicht} verfügbar, kann der
    Verweis über eine \emph{Fußnote} erledigt werden.

    \item Zwei \emph{aufeinanderfolgende} Sätze tragen \emph{niemals} die gleiche Referenz.

  \end{itemize}
\end{frame}

%%%%%%%%%%%%%%%%%%%%%%%%%%%%%%%%%%%%%%%%%%%%%%%%%%%%%%%%%%%%%%%%%%%%%%%%%%%%%%%%%%%%%
\begin{frame}
  \frametitle{Anmerkungen zu den Entwürfen}
  \framesubtitle{Aufbau}

  \begin{itemize}
    \item Wissenschaftliche Arbeiten wirken etwas ``senil'' (weil repetitiv) im
    Aufbau. Umgangssprachlich:
    \begin{itemize}
      \item Die \emph{Einleitung} spricht darüber, was man tun wird,
      \item der \emph{Hauptteil} erklärt was man tut,
      \item die \emph{Evaluation} versucht zu zeigen, dass das was man tut plausibel
      ist und
      \item der \emph{Schluss} spricht darüber was man getan hat.
    \end{itemize}
    
    \item Generell \emph{leiten Abschnitte} ineinander \emph{über} und \emph{referenzieren} sich (in
    moderatem Umfang) \emph{gegenseitig} innerhalb der Arbeit.
    
    \item Am \emph{Ende} der Einleitung befindet sich ein Absatz, der die \emph{kommenden Abschnitte}
    der Arbeit sowie deren Aufbau \emph{zusammenfasst}.
    
    \item Unterüberschriften treten nur auf, wenn es auf \emph{gleicher Ebene} mindestens noch
    \emph{eine weitere} Überschrift gibt.
    
  \end{itemize}
\end{frame}

\begin{frame}
  \frametitle{Anmerkungen zu den Entwürfen}
  \framesubtitle{Timeline}
  
  \begin{itemize}
    \item Fällig \emph{2019-06-13}
    \item Wer \emph{keinen} Entwurf vorlegt, nimmt \emph{nicht} am Review teil
  \end{itemize}
\end{frame}

%%%%%%%%%%%%%%%%%%%%%%%%%%%%%%%%%%%%%%%%%%%%%%%%%%%%%%%%%%%%%%%%%%%%%%%%%%%%%%%%%%%%%%
%\begin{frame}
%  \frametitle{Review}
%  \framesubtitle{Eckdaten}
%
%  \begin{itemize}
%    \item Beurteilung von \emph{zwei} fremden Seminararbeiten
%    \item Abgabe \emph{Mittwoch}, \emph{21.12.2016}, \emph{14:00 Uhr}
%    \medskip
%    \item Zwei Arbeiten \emph{r0.pdf}, \emph{r1.pdf} per git Repository (heute abend)
%    \item Abgabe via git, Dateien \emph{r0-reviewed.pdf} und \emph{r1-reviewed.pdf}
%    \item Maximal \emph{eine A4-Seite} pro Arbeit
%    \item Sprache \emph{deutsch}
%    \item Wer \emph{keinen} Entwurf vorlegt, nimmt \emph{nicht} am Review teil
%  \end{itemize}
%\end{frame}
%
%\begin{frame}
%  \frametitle{Review}
%  \framesubtitle{Kriterien für ein gutes Review}
%  
%  Ein gutes Review enthält (wenn zutreffend):
%  \begin{itemize}
%    \item Eine \emph{Zusammenfassung} der Arbeit aus Sicht des Reviewers
%    \item Hinweise auf \emph{formale Fehler} (Einheitlichkeit Quellen, Lesbarkeit
%    Grafiken, ...)
%    \item Konkrete \emph{inhaltliche} Kritik (Probleme aufzeigen, Verbesserungsvorschläge)
%    \item Anregungen für \emph{künftige} Arbeitsschritte
%    \item Anmerkungen zu \emph{Orthographie} und \emph{Grammatik}
%  \end{itemize}
%  \medskip
%  Vorsicht vor \emph{``Rekursion''}: Oftmals werden genau die Fehler kritisiert, die
%  der Verfasser selbst gemacht hat.
%\end{frame}

\end{document}