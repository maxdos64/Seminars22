\documentclass{i20lecture}

\subtitle{Session 2}
\usepackage{amssymb}

\newcommand{\done}{\makebox[0pt][l]{$\square$}\raisebox{.15ex}{\hspace{0.1em}$\checkmark$}}
\begin{document}

\frame{\titlepage}

%%%%%%%%%%%%%%%%%%%%%%%%%%%%%%%%%%%%%%%%%%%%%%%%%%%%%%%%%%%%%%%%%%%%%%%%%%%%%%%%%%%%%
\begin{frame}
  \frametitle{Das Verfassen eines Abstracts}
  \framesubtitle{Inhalt}

  Ein Abstract ist eine verkürzte Darstellung einer wissenschaftlichen Arbeit,
  welche die Kernideen der Arbeit sowie deren Ergebnisse widergibt.

  \medskip
  Wesentliche Inhalte eines Abstracts:
  \begin{itemize}
    \item Motivation/Problemstellung (2 - 3 Sätze)
    
    Inwiefern ist das betrachtete Problem sowie seine Lösung relevant? Welche
    Problemstellung wird betrachtet?

    \item Lösungsansatz (Zweitlängster Abschnitt)

    Welche Technik/Idee wurde zur Lösung des Problems verwendet?

    \item Ergebnisse (Wichtigster und längster Abschnitt)

    Was ist das erzielte Ergebnis?
    
    \item Schlussfolgerungen (Schlüsselbotschaft)

    Welche Konsequenzen welcher Größenordnung zieht das Ergebnis nach sich?

  \end{itemize}
\end{frame}

%%%%%%%%%%%%%%%%%%%%%%%%%%%%%%%%%%%%%%%%%%%%%%%%%%%%%%%%%%%%%%%%%%%%%%%%%%%%%%%%%%%%%
\begin{frame}
  \frametitle{Das Verfassen eines Abstracts}
  \framesubtitle{Inhaltliche Kriterien}
  
  \begin{itemize}
    \item Auch verständlich für den erweiterten Leserkreis (``Informatiker'')
    \item Selbsterklärend, in sich schlüssig
    \item Limitierungen werden hervorgehoben
  \end{itemize}

\end{frame}

%%%%%%%%%%%%%%%%%%%%%%%%%%%%%%%%%%%%%%%%%%%%%%%%%%%%%%%%%%%%%%%%%%%%%%%%%%%%%%%%%%%%%
\begin{frame}
  \frametitle{Das Verfassen eines Abstracts}
  \framesubtitle{Formale Kriterien}

  \begin{itemize}
    \item Platzierung vor der Einleitung
    \item Länge zwischen 150 und 200 Wörtern
    \item Keine Zitate
    \item Vermeiden von ``Weichmachern'' wie ``ziemlich'', ``meistens'',
    ``relativ''
    \item Keine Informationen, die nicht in der Arbeit selbst auch erscheinen
  \end{itemize}

\end{frame}

%%%%%%%%%%%%%%%%%%%%%%%%%%%%%%%%%%%%%%%%%%%%%%%%%%%%%%%%%%%%%%%%%%%%%%%%%%%%%%%%%%%%%
\begin{frame}
  \frametitle{Das Verfassen eines Abstracts}
  \framesubtitle{Abgrenzung zur Einleitung}
  
  \begin{itemize}
    \item Abstract fasst die gesamte Arbeit zusammen, inklusive aller
    Ergebnisse.
    \begin{itemize}
      \item Erinnerung: Motivation/Problemstellung, Lösungsansatz, Ergebnisse, Schlussfolgerungen
    \end{itemize}
    \item Die Einleitung führt auf das Problem hin und motiviert.
    \begin{itemize}
      \item Hintergrund, Motivation, Problemstellung, Arbeitshypothese, 
    \end{itemize}
    \bigskip
    \item Weiterführende Informationen:
    \begin{itemize}
      \item \url{https://users.ece.cmu.edu/~koopman/essays/abstract.html}
      \item \url{http://www.adelaide.edu.au/writingcentre/learning_guides/learningGuide_writingAnAbstract.pdf}
      \item \url{https://www.ncbi.nlm.nih.gov/pmc/articles/PMC3136027/}
      \end{itemize}
  \end{itemize}

\end{frame}

%%%%%%%%%%%%%%%%%%%%%%%%%%%%%%%%%%%%%%%%%%%%%%%%%%%%%%%%%%%%%%%%%%%%%%%%%%%%%%%%%%%%%

%\begin{frame}
%  \frametitle{Das Verfassen eines Abstracts}
%  \framesubtitle{Informatives Abstract: Beispiel}
%
%  ``Metalinguistic awareness contributes to effective writing at university. Writing
%is a meaning-making process where linguistic, cognitive, social and creative
%factors are at play. University students need to master the skills of academic
%writing not only for getting their degree but also for their future career. It
%is also significant for lecturers to know who our students are, how they think
%and how we can best assist them. This study examines first-year undergraduate
%Australian and international engineering students as writers of academic texts
%in a multicultural setting at the University of Adelaide. A questionnaire and
%interviews were used to collect data about students’ level of metalinguistic
%awareness, their attitudes toward, expectations for, assumptions about and
%motivation for writing. The preliminary results of the research show that
%students from different cultures initially have different concepts about the
%academic genres and handle writing with different learning and writing styles,
%but those with a more developed metal anguage are more confident and motivated.
%The conclusion can also be drawn that students’ level of motivation for academic
%writing positively correlates with their opinion about themselves as writers.
%Following an in-depth multi-dimensional analysis of preliminary research
%results, some recommendations for writing instruction will also be presented.''
%  
%\end{frame}

\end{document}