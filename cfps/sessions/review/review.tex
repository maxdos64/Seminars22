
%% see http://latex-beamer.sourceforge.net/
%% idea contributed by H. Turgut Uyar
%% template based on a template by Till Tantau
%% this template is still evolving - it might differ in future releases!

\documentclass[xcolor={usenames,dvipsnames}]{beamer}
\usepackage{booktabs}
\usepackage{ragged2e}
\hypersetup{colorlinks,linkcolor=,urlcolor=}


% THEME
% =========================================================
%\usetheme{Boadilla}
\setbeamertemplate{navigation symbols}{}
\setbeamercolor{normal text}{fg=white,bg=black!90}
\setbeamercolor{structure}{fg=white}
\setbeamercolor{item projected}{use=item,fg=white,bg=item.fg!35}
\setbeamercolor*{palette primary}{use=structure,fg=structure.fg}
\setbeamercolor*{palette secondary}{use=structure,fg=structure.fg!95!black}
\setbeamercolor*{palette tertiary}{use=structure,fg=structure.fg!90!black}
\setbeamercolor*{palette quaternary}{use=structure,fg=structure.fg!95!black,bg=black!80}
\setbeamercolor*{framesubtitle}{fg=white}
\setbeamercolor*{block title}{parent=structure,bg=black!60}
\setbeamercolor*{block body}{fg=black,bg=black!10}
\setbeamercolor*{block title alerted}{parent=alerted text,bg=black!15}
\setbeamercolor*{block title example}{parent=example text,bg=black!15}

\setbeamercolor{mybox}{fg=white,bg=GreenYellow}
%\useinnertheme[shadow]{rounded}
% =========================================================

% DEFINE COLORS
% =========================================================
\definecolor{darkred}{RGB}{223,63,00}
\definecolor{brightred}{RGB}{255,127,00}
% =========================================================

% Templates
% =========================================================
%\setbeamertemplate{itemize subitem}[triangle]
% =========================================================

% SET COLORS
% =========================================================
\setbeamercolor{normal text}{fg=white,bg=black!90}
\setbeamercolor{structure}{fg=white}
\setbeamercolor{item projected}{use=item,fg=white,bg=item.fg!35}
\setbeamercolor*{palette primary}{use=structure,fg=structure.fg}
\setbeamercolor*{palette secondary}{use=structure,fg=structure.fg!95!black}
\setbeamercolor*{palette tertiary}{use=structure,fg=structure.fg!90!black}
\setbeamercolor*{palette quaternary}{use=structure,fg=structure.fg!95!black,bg=black!80}
\setbeamercolor*{framesubtitle}{fg=white}
\setbeamercolor*{block title}{parent=structure,bg=black!60}
\setbeamercolor*{block body}{fg=black,bg=black!10}
\setbeamercolor*{block title alerted}{parent=alerted text,bg=black!15}
\setbeamercolor*{block title example}{parent=example text,bg=black!15}
% =========================================================


% FONTS
% =========================================================
\setbeamerfont{alerted text}{series=\bfseries}
% =========================================================


% PACKAGES
% =========================================================
\usepackage[english]{babel}
\usepackage[utf8]{inputenc}
\usepackage{DejaVuSansMono}
\usepackage[T1]{fontenc}
\usepackage{tikz}
% =========================================================

% METADATA
% =========================================================
\title[CFPS WS~22]{CFPS --- WS 2022}

\subtitle{Seminar}

\author[M. Tschirschnitz]
{
	Maximilian Tschirschnitz
}

\institute[Chair I20, TUM]
{
	Lehrstuhl f\"ur Sicherheit in der Informatik / I20 \\
	Prof.\ Dr.\ Claudia Eckert\\
	Technische Universität München
}

\date{\today}
% =========================================================



% If you have a file called "university-logo-filename.xxx", where xxx
% is a graphic format that can be processed by latex or pdflatex,
% resp., then you can add a logo as follows:

% \pgfdeclareimage[height=0.5cm]{university-logo}{university-logo-filename}
% \logo{\pgfuseimage{university-logo}}



% Delete this, if you do not want the table of contents to pop up at
% the beginning of each subsection:
%\AtBeginSubsection[]
%{
%\begin{frame}<beamer>
%\frametitle{Outline}
%\tableofcontents[currentsection,currentsubsection]
%\end{frame}
%}

% If you wish to uncover everything in a step-wise fashion, uncomment
% the following command:

%\beamerdefaultoverlayspecification{<+->}

\begin{document}

\begin{frame}
\titlepage
\end{frame}

\begin{frame}
	\frametitle{Review - Structure}

			\begin{enumerate}
				\item Summary
				\item Strength
				\item Weakness
				\item Comments
			\end{enumerate}

			\vspace{1cm}

			\begin{itemize}
				\item<2> \emph{\alert{$\sim$ About one page}}
			\end{itemize}
\end{frame}

\begin{frame}
	\frametitle{Summary}

	\begin{columns}
		\begin{column}{0.5\linewidth}
			\begin{itemize}
				\item \alert{Shortly} summarize the contents
				\item What is the topic?
				\item What is new about it?
				\item Which new approaches were developed?
				\item \emph{$\sim$ About one paragraph}
			\end{itemize}
		\end{column}
		\begin{column}{0.49\linewidth}
			\tiny
			\justifying
			The paper introduces an alternative to the traditional Unix ptrace
			facility and an implementation of this alternative as a Linux loadable
			kernel module and a GDB server interfacing to the module.  The
			alternative, called plutonium-dbg, is meant to address some shortcomings
			of ptrace, but ends up having its own different shortcomings, which the
			paper acknowledges and discusses.

			Overall, this is a fine topic for research, a practically usable
			implementation, and a decent paper.  There's no reason not to accept
			this paper.
		\end{column}
	\end{columns}
\end{frame}

\begin{frame}
	\frametitle{Why Do I Have to Summerize the Paper?}

	\begin{itemize}
		\item Helps you understand the paper
		\item Shows the author that you understood the paper
		\item \alert{Or that you did not!}
		\item May show the reason for good or bad feedback
		\item Helps to detect what is unclear
		\item Shows the perspective of the reader
	\end{itemize}
\end{frame}

\begin{frame}
	\frametitle{Strength}

	\begin{columns}
		\begin{column}{0.5\linewidth}
			\begin{itemize}
				\item Highlight the strength of the paper
				\item What did you like?
				\item Content related
				\item Focus on outstanding points
			\end{itemize}
		\end{column}
		\begin{column}{0.49\linewidth}
			\tiny
			Strengths:

			+ The paper is well written and easy to follow and the authors guide the
			reader through the different challenges

			+ They address an important problem and they provide a solution. We as a
			community lack of specific tools for reversing on Linux compared to all
			the available programs for Windows

			+ This solution is released on Github

			+ They provide a really nice summary of all the known techniques to
			detect a debugger on Linux. This is the most completed list I know

			+ The tool is compatible with the GDB protocol

			+ Nice use of the kprobes and uprobes for addressing all the technical
			challenges

		\end{column}
	\end{columns}
\end{frame}

\begin{frame}
	\frametitle{Weakness}

	\begin{columns}
		\begin{column}{0.5\linewidth}
			\begin{itemize}
				\item Highlight the Weakness of the paper
				\item What did you dislike?
				\item Content related
				\item \alert{Focus on issues that teach the author something}
				\item Do \textbf{not} list spelling or grammar errors
			\end{itemize}
		\end{column}
		\begin{column}{0.49\linewidth}
			\tiny
			Weaknesses:

			- The tool supports only x86\_64 binaries

			- They insist plutonium-dbg is important for malware analysis but as
			Cozzi et al.[1] pointed out x86\_64 is not the most common architecture
			for Linux malware.

			- Maybe there are other instructions or events that as a side effect
			push on the stack RFLAGS so the solution proposed for the pushf is not solid

			- The authors do not discuss at all how common are the mentioned evasive
			tricks. Table XIII of [1] gives an idea.

			- The evaluation part does not mention which binaries were used for the
			experiments. Real malware samples? Quick pocs? Please specify.
		\end{column}
	\end{columns}
\end{frame}

\begin{frame}
	\frametitle{Comments}

	\begin{columns}
		\begin{column}{0.5\linewidth}
			\begin{itemize}
				\item Text-related comments
				\item Target to improve the document
				\item Specific suggestions for improvement
			\end{itemize}
		\end{column}
		\begin{column}{0.49\linewidth}
			\tiny
			Comments:

			After headline of 1 and 3: A short introduction to the chapter would
			be good.

			In 1.3: The groups should be mentioned. No detailed explanation needed, a short
			enumeration is enough.

			In 2: A brief overview about the content of LO! and CONFUSE would help. It is
			unclear how exactly they are related to this paper and what their results are.

			In 4.2: "Opaque Predicates are boolean functions that always return a fixed
			value regardless of their input." Example of missing citation mentioned above.
			Who introduces opaque predicates? Where is this definition taken from?

			In 5.2: "This is a FunctionPass." It is unclear what is referred to by "this"

			In 5.3: Typo at "after the second variable and insert the perdicate".
			"perdicate" should be changed to "predicate".
		\end{column}
	\end{columns}
\end{frame}

\begin{frame}
	\frametitle{Your Task}

	\begin{itemize}
		\item \alert{Carefully} read through the two papers we provide you with
		\item Write reviews for both (about one page each)
		\item Hand in until \alert{Dec, 20th} latest!
		\item Improve your draft based on reviews you receive afterwards
	\end{itemize}
\end{frame}

\begin{frame}
	\frametitle{Don't be shy}
	\begin{itemize}
		\item We do not tell you who is the author (you might know)
		\item We do not tell the author who has written the review
	\end{itemize}

\end{frame}

\begin{frame}
	\frametitle{Writing an Abstract}
	\framesubtitle{Content}

	Your Abstract is a short version of your paper.
	It should reflect all core ideas and results.

	\medskip
			\begin{itemize}
				\item Motivation/Problem (2 - 3 sentences)

					How is your problem/solution relevant. What is the problem?
				\item Solution/Analysis Approach (second largest part)

					Which approach/technique was used to address this problem.
				\item Results

					Most important part.
				\item Conclusion

					What follows from your results/consequences?
			\end{itemize}
\end{frame}

\begin{frame}
	\frametitle{Writing an Abstract}
	\framesubtitle{Style}

	\begin{itemize}
		\item Easy to comprehend/follow for a broad audience.
		\item Closed in itself and self-explaining.
		\item Be honest and present limitations and problems truthfully.
	\end{itemize}

\end{frame}

\begin{frame}
	\frametitle{Writing an Abstract}
	\framesubtitle{Format}

	\begin{itemize}
		\item Between 150 and 200 words.
		\item No citations/references (usually).
		\item Concrete language (avoid words like `mostly', `relatively', etc.).
		\item Only content from your work.
	\end{itemize}

\end{frame}

\begin{frame}
	\frametitle{Writing an Abstract}
	\framesubtitle{Not an Introduction}

	\begin{itemize}
		\item Covers the whole paper.
		\item Introduction leads towards the remainder of the work.
		\item Introduction can be a bit more `scenic'.
		\item Abstract needs to get your point delivered in a snap (advertisement).
	\end{itemize}

	Further Reading:
	\begin{itemize}
		\item \url{https://users.ece.cmu.edu/~koopman/essays/abstract.html}
		\item \url{http://www.adelaide.edu.au/writingcentre/learning_guides/learningGuide_writingAnAbstract.pdf}
		\item \url{https://www.ncbi.nlm.nih.gov/pmc/articles/PMC3136027/}
	\end{itemize}

\end{frame}

\begin{frame}
	\frametitle{Questions?}
\end{frame}
\end{document}
