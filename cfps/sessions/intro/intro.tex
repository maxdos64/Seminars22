\documentclass{i20lecture}

\subtitle{Session 1}
\usepackage{amssymb}

\newcommand{\done}{\makebox[0pt][l]{$\square$}\raisebox{.15ex}{\hspace{0.1em}$\checkmark$}}
\begin{document}

\frame{\titlepage}

%%%%%%%%%%%%%%%%%%%%%%%%%%%%%%%%%%%%%%%%%%%%%%%%%%%%%%%%%%%%%%%%%%%%%%%%%%%%%%%%%%%%%
\begin{frame}
  \frametitle{Timeline}

  \begin{center}
    % \begin{tikzpicture}[scale=.75]
    %   \path[|-|] (0, 0) edge node[pos=.5, fill=white] {\footnotesize Seminar \texttt{cfps21s}} (18, 0);
    %   \path[|-|, shorten > = .5pt, shorten < = .0pt, ] ( 0.00, 0.75) edge node[pos=.5, fill=white] {\footnotesize \alert{I}}    ( 1, 0.75);
    %   \path[|-|, shorten > = .5pt, shorten < = .5pt, ] ( 1.00, 0.75) edge node[pos=.5, fill=white] {\footnotesize \alert{II}}   ( 2, 0.75);
    %   \path[|-|, shorten > = .5pt, shorten < = .5pt, ] ( 2.00, 0.75) edge node[pos=.5, fill=white] {\footnotesize \alert{III}}  (10, 0.75);
    %   \path[|-|, shorten > = .5pt, shorten < = .5pt, ] (10.00, 0.75) edge node[pos=.5, fill=white] {\footnotesize \alert{IV}}   (11, 0.75);
    %   \path[|-|, shorten > = .5pt, shorten < = .5pt, ] (11.00, 0.75) edge node[pos=.5, fill=white] {\footnotesize \alert{V}}    (16, 0.75);
    %   \path[|-|, shorten > = .5pt, shorten < = .5pt, ] (16.00, 0.75) edge node[pos=.5, fill=white] {\footnotesize \alert{VI}}   (17, 0.75);
    %   \path[|-|, shorten > = .0pt, shorten < = .5pt, ] (17.00, 0.75) edge node[pos=.5, fill=white] {\footnotesize \alert{VII}}  (18, 0.75);
    % \end{tikzpicture}
    % \vfill
    \scalebox{.9}{
    \begin{tabular}{lp{5.5cm}ll}
      \toprule
      Phase       & Description            			& Creates Deliverable                     & Due Date \\
      \midrule
      \alert{P}	& Choose a topic          			& Citation graph                 			& \alert{18.10.2022} \\
  		\alert{I}   & Deep dive \& Concept    			& Identify research questions      & \alert{25.10.2022} \\
  		\alert{II}  & Refine concept  					& A well structured idea for your paper   & \alert{08.11.2022} \\
      \alert{III} & Writing  								& Paper draft (review version) 				& \alert{13.12.2022} \\
      \alert{IV}  & Reviewing \& writing           & Your reviews       							& \alert{20.12.2022} \\
      \alert{VI}  & Presentation deadline			  	& Presentation ready  							& \alert{30.01.2023} \\
      \midrule
      --          & End of Lecture Period   			& Paper camera ready                                   & \alert{11.02.2023} \\
      \bottomrule
    \end{tabular}
    }
  \end{center}
\end{frame}

\begin{frame}
	\frametitle{Goal of this Course}

	We want you to develop your skills as a member of the \alert{scientific community}:
	\vspace{0.5cm}
	\begin{itemize}
		\item Surveying skills (Related Work).
		\item Ability to conceptualize abstract concepts for yourself.
		\item Ability to adjust such concepts.
		\item Ability to present your thoughts to others in clarity.
	\end{itemize}

	\vspace{1cm}
	$\Rightarrow$ \textbf{Show us that you posses those skills, by presenting a representative work!}
\end{frame}

\begin{frame}
	\frametitle{For Example}
	Some paper types that would be perfectly suited as demonstrator of your amazing research skills:
	\vspace{0.5cm}
	\begin{itemize}
		\item Simplified representation of existing knowledge (simplify for the next reader).
		\item Reproduction of an experiment and interpretation of results.
		\item Creative combinations of existing techniques.
		\item Systemization of Knowledge and discussion.
	\end{itemize}
	\vspace{1cm}
	\center{
	\textbf{Your paper does not need to present groundbreaking innovation!}\newline(ofc cool if it does)
	}
\end{frame}

% \begin{frame}
% 	\frametitle{The ``Ampel'' System}
% 
% 	\center{\textbf{You are responsible for the quality of your work. We are here to help you!}}
% 	\vspace{0.5cm}
% 	\begin{itemize}
% 		\item We promise you our personal best guidance and support.
% 		\begin{itemize}
% 			\item We are experienced in making mistakes ! :-)
% 		\end{itemize}
% 		\item After every exercise/personal talk you receive a \textbf{Ampel signal} $\in \{\textcolor{red}{red}, \textcolor{orange}{yellow}, \textcolor{green}{green}\}$.
% 		\item Our experience: A good grade is \textbf{correlated with but not implied by} green lights. 
% 		\begin{itemize}
% 			\item You may get a 1.0 with only red lights (unlikely) or a 5.0 with green lights (even more unlikely)
% 		\end{itemize}
% 		\item You may choose at any time whether you receive your signal in confidence or publicly.
% 	\end{itemize}
% 
% \end{frame}


\begin{frame}
  \frametitle{Orga}
  \begin{itemize}
	  \item Meeting (about) every \alert{2} weeks in presence. \textbf{The TUM calendar counts!}
	  \item Regularly: Complete provided exercises, and receive feedback from us and others.
	  \item We will have at least one round of personal meetings with us and each of you. (When?)
	  \item Issues?
  \end{itemize}
\end{frame}


\begin{frame}[fragile]
	\frametitle{Help with Literature}
  \begin{itemize}
      \item \url{https://scholar.google.com}
      \item \url{https://semanticscholar.org}
	  \item \url{https://dblp.uni-trier.de}
      \item \url{https://arxiv.org}
      \item Get around paywalls using
          \url{https://login.eaccess.ub.tum.de/login} or bookmarklet:
          \begin{lstlisting}
javascript:void(location.href='https://eaccess.ub.tum.de/login?url='+location.href)
          \end{lstlisting}
      \item Researchers' homepages can be \alert{valuable}! (source code, raw data, instructions, technical information, ...)
  \end{itemize}
\end{frame}

\begin{frame}
	\frametitle{Buddy System}

	\center{\textbf{``Hey... you wanna be buddys ?''}}
\end{frame}



\begin{frame}
	\frametitle{Next Exercise} 
	\center{
		\textbf{Identify points of interests for your paper concept}

		For this we ask you to execute the following iterative process.
		}
\end{frame}

\begin{frame}
		
		\small
		\textbf{1. Go through your catalog of papers (\alert{Make notes!}) and ask yourself questions like}
		\begin{enumerate}[label=(\alph*)]
			\item{What are remaining research questions, what was hard to understand? \textbf{What annoyed you?} \alert{Provide context to your answers!}}
			\item{Have those results been reproduced? If not, why not ?}
			\item{Which works or results should be compared with each other ? What would such a discussion yield?}
			\item{Why was an attack possible ? Is there an underlying systematic problem? What was affected? What can be done now ?}
		\end{enumerate}

		\vspace{0.5cm}

		\textbf{2. Try to resolve those questions (e.g.)}
		\begin{enumerate}[label=(\alph*)]
			\item{Are there better explanations in other papers ? Has future work already been addressed ?}
			\item{Are there already reproductions of this aspect ? If so, do you agree with the methodology ?}
			\item{What comparisons where drawn in other works on this topic?}
			\item{How did the community react to this attack ? Are there follow up-publications?}
		\end{enumerate}

		\textbf{3. Repeat on the new material until you are left with a large \alert{collection of unresolved questions.}}
\end{frame}

\begin{frame}
	\frametitle{Next Exercise} 

\center{
\textbf{Be ready to present those possible research questions to the course.}
}

\end{frame}

\begin{frame}
	\frametitle{The End (for today)}
	\begin{center}
		\Large Questions?
	\end{center}
\end{frame}

\end{document}
