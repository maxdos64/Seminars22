
%% see http://latex-beamer.sourceforge.net/
%% idea contributed by H. Turgut Uyar
%% template based on a template by Till Tantau
%% this template is still evolving - it might differ in future releases!

\documentclass[xcolor={usenames,dvipsnames}]{beamer}
\usepackage{booktabs}


% THEME
% =========================================================
%\usetheme{Boadilla}
\setbeamertemplate{navigation symbols}{}
\setbeamercolor{normal text}{fg=white,bg=black!90}
\setbeamercolor{structure}{fg=white}
\setbeamercolor{item projected}{use=item,fg=white,bg=item.fg!35}
\setbeamercolor*{palette primary}{use=structure,fg=structure.fg}
\setbeamercolor*{palette secondary}{use=structure,fg=structure.fg!95!black}
\setbeamercolor*{palette tertiary}{use=structure,fg=structure.fg!90!black}
\setbeamercolor*{palette quaternary}{use=structure,fg=structure.fg!95!black,bg=black!80}
\setbeamercolor*{framesubtitle}{fg=white}
\setbeamercolor*{block title}{parent=structure,bg=black!60}
\setbeamercolor*{block body}{fg=black,bg=black!10}
\setbeamercolor*{block title alerted}{parent=alerted text,bg=black!15}
\setbeamercolor*{block title example}{parent=example text,bg=black!15}

\setbeamercolor{mybox}{fg=white,bg=GreenYellow}
%\useinnertheme[shadow]{rounded}
% =========================================================

% DEFINE COLORS
% =========================================================
\definecolor{darkred}{RGB}{223,63,00}
\definecolor{brightred}{RGB}{255,127,00}
% =========================================================

% Templates
% =========================================================
%\setbeamertemplate{itemize subitem}[triangle]
% =========================================================

% SET COLORS
% =========================================================
\setbeamercolor{normal text}{fg=white,bg=black!90}
\setbeamercolor{structure}{fg=white}
\setbeamercolor{item projected}{use=item,fg=white,bg=item.fg!35}
\setbeamercolor*{palette primary}{use=structure,fg=structure.fg}
\setbeamercolor*{palette secondary}{use=structure,fg=structure.fg!95!black}
\setbeamercolor*{palette tertiary}{use=structure,fg=structure.fg!90!black}
\setbeamercolor*{palette quaternary}{use=structure,fg=structure.fg!95!black,bg=black!80}
\setbeamercolor*{framesubtitle}{fg=white}
\setbeamercolor*{block title}{parent=structure,bg=black!60}
\setbeamercolor*{block body}{fg=black,bg=black!10}
\setbeamercolor*{block title alerted}{parent=alerted text,bg=black!15}
\setbeamercolor*{block title example}{parent=example text,bg=black!15}
% =========================================================


% FONTS
% =========================================================
\setbeamerfont{alerted text}{series=\bfseries}
% =========================================================


% PACKAGES
% =========================================================
\usepackage[english]{babel}
\usepackage[utf8]{inputenc}
\usepackage{DejaVuSansMono}
\usepackage[T1]{fontenc}
\usepackage{tikz}
% =========================================================

% METADATA
% =========================================================
\title[Reverse Engineering WS~16]{Reverse Engineering --- SS 2020}

\subtitle{Seminar}

\author[J. Kirsch]
{
	Fabian Franzen,
	Ludwig Peuckert
}

\institute[Chair I20, TUM]
{
	Lehrstuhl f\"ur Sicherheit in der Informatik / I20 \\
	Prof.\ Dr.\ Claudia Eckert\\
	Technische Universität München
}

\date{\today}
% =========================================================



% If you have a file called "university-logo-filename.xxx", where xxx
% is a graphic format that can be processed by latex or pdflatex,
% resp., then you can add a logo as follows:

% \pgfdeclareimage[height=0.5cm]{university-logo}{university-logo-filename}
% \logo{\pgfuseimage{university-logo}}



% Delete this, if you do not want the table of contents to pop up at
% the beginning of each subsection:
%\AtBeginSubsection[]
%{
%\begin{frame}<beamer>
%\frametitle{Outline}
%\tableofcontents[currentsection,currentsubsection]
%\end{frame}
%}

% If you wish to uncover everything in a step-wise fashion, uncomment
% the following command:

%\beamerdefaultoverlayspecification{<+->}

\begin{document}

\begin{frame}
\titlepage
\end{frame}

%\begin{frame}
%\frametitle{Overview}
%\tableofcontents
%% You might wish to add the option [pausesections]
%\end{frame}

\begin{frame}
	\frametitle{Preface}

	Sorry for Corona :/

	\vspace{0.5cm}
	We have no clue how the semester develops, we are still working on a detailed schedule.

	\pause
	\vspace{0.5cm}
	Most likely a visit to Fraunchiemsee will \alert{not} be possible :///

	\pause
	\vspace{0.2cm}
	{\scriptsize (but we will check again from time to time)}
	
	\pause
	\vspace{0.2cm}
	Please join our online meeting on \alert{Wed, 22th April 2020}
\end{frame}

\begin{frame}[label=process]
	\frametitle{Process}
	\begin{itemize}
		\alert<2->{\item Phase \alert{I}: Select a \alert{topic}}
		\alert<3->{\item Phase \alert{II}: Find \alert{literature}}
		\item Phase \alert{III}: Do your \alert{reading / experiments / programming}
		\item Phase \alert{IV}: \alert{Writing} phase I
		\item Phase \alert{V}: \alert{Peer review}
		\item Phase \alert{VI}: \alert{Writing} phase II
		\item Phase \alert{VII}: Final \alert{talks}
	\end{itemize}
\end{frame}

\begin{frame}
	\frametitle{Course goal}

	\textbf{Teaching you scientific writing!}

	\vspace{1cm}

	You have to write a paper
\end{frame}

\begin{frame}
	\begin{tikzpicture}[overlay, remember picture]
		\node[anchor=center] at (current page.center) {\huge Literature Research};
	\end{tikzpicture}
\end{frame}

\begin{frame}
	\frametitle{Where to look?}

	\textbf{Non-Scientific}
	\begin{itemize}
		\item Google\footnote{or of course any other search engine of your choice} (Blog Posts, Mailing lists, Software Repos on GitHub)
	\end{itemize}

	\vspace{1cm}
	\textbf{Scientific}
	\begin{itemize}
		\item Library (Books, E-Books)
		\item Google Schoolar (Research Papers)
		\item SemanticScholar
		\item DBLP
	\end{itemize}
\end{frame}

\begin{frame}
	\begin{tikzpicture}[overlay, remember picture]
		\node[anchor=center] at (current page.center) {\huge Live Demo};
	\end{tikzpicture}
\end{frame}

\begin{frame}[fragile]
	\frametitle{How to bybass the scientific paywall?}

	\begin{enumerate}
		\item Use the \alert{bookmarklet} of our university library:
		\vspace{0.5em}
		\tiny
			\begin{verbatim}
			javascript:void(location.href='https://eaccess.ub.tum.de/login?url='+location.href)
			\end{verbatim}
		\normalsize
		\item Look on the webpage of the authors or their university
	\end{enumerate}
	
	\vspace{1cm}
	Due to the Covid-19 crisis: ACM Digital Library is open access right now!
\end{frame}

\begin{frame}
	\frametitle{Abuse the related work section}

	\begin{itemize}
		\item Almost every paper in CS contains a \emph{Related Work} section
			\begin{itemize}
			\item Makes it easier for the reader (and the reviewers) to judge academic novelty
			\item May contain \alert{other} interesting reads!
			\end{itemize}
		\item Check the References in the end about other interesting citations.
	\end{itemize}

	\vspace{1cm}

	\textbf{Thumbrule in academia:} The more citations a paper has, the more important it is.\footnote{Of course this rule is flawed in multiple ways, but it is a thumbrule}
\end{frame}

\end{document}
